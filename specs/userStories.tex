\documentclass[oneside]{article}

\usepackage[utf8]{inputenc}
\usepackage{color} %Pacote utilizado para que seja possível adicionar cores ao texto
\usepackage{hyperref} %Pacote utilizado para adicionar bookmarks no PDF
\usepackage{fancyhdr}

\begin{document}
\title{Sistema de Máquina de Distribuição de Alimentos  \\
        \large }
\addcontentsline{toc}{chapter}{Título --- Projeto para Disciplina de SIG}
\author{Eduardo Dorneles Ferreira de Souza}
%\date{2 de Julho de 2016}
\maketitle

\pagestyle{plain}

\section{User Stories}

\begin{center}
    \begin{tabular}{| l | p{11cm} |}
        \hline
        User Story & Texto\\ \hline
        RF001 & Haverá um módulo usuário para fornecer os alimentos\\ \hline
        & Haverá um módulo de manutenção\\ \hline
        & [USUÁRIO] O usuário deve inserir cédulas (de 2 à 10 reais) ou moedas
            (0,50 centavos ou 1 real), selecionar um alimento desejado
            a máquina deverá fornecê-lo se estiver com validade junto do troco\\ \hline
        & [USUÁRIO] 6 tipos de alimentos a serem fornecidos\\
        & [USUÁRIO] Os diferentes tipos de alimentos serão separados por bandejas, \\ \hline
        & [USUÁRIO] O troco ao cliente deverá ser entregue do maior valor ao menor \\ \hline
                    (ex: se ele deve receber 4 reais de troco, receberá 2 cédulas
                    de 2 reais) \\ \hline
        & [MANUTENÇÃO] O operador de máquina se autentica para reabastecer com
                        alimentos\\ \hline
        & [MANUTENÇÃO] O mesmo tipo de produto pode ter um bandeja separada por
            conta da validade\\ \hline
        & [MANUTENÇÃO] A bandeja de troco deve ser reabastecida e é separada
                        do cofre das vendas \\
        \hline
    \end{tabular}
\end{center}

\clearpage

\end{document}
